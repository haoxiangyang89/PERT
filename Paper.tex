\documentclass[11pt]{article}
\usepackage[small]{titlesec}
\usepackage[top = 0.66in,textwidth = 6.5in, textheight=9.1in]{geometry}

\usepackage{amsmath}
\usepackage{graphicx}
\usepackage{latexsym}
\usepackage{color}
\usepackage{amssymb}
\usepackage{tabularx}
\usepackage{fancyhdr}
\usepackage{verbatim}
\usepackage{multirow}
\usepackage{framed}
\usepackage{natbib}
\usepackage{float, subfig}
\usepackage{enumitem}
\usepackage{mathtools}
\usepackage{mathrsfs}
\usepackage{amsfonts}
\usepackage{listings}
\usepackage{amsthm}
\usepackage{grffile}
\usepackage{sidecap}
\usepackage{pbox}
\usepackage{algorithm}
\usepackage[noend]{algpseudocode}

\def\qed{\hfill{\(\vcenter{\hrule height1pt \hbox{\vrule width1pt height5pt
     \kern5pt \vrule width1pt} \hrule height1pt}\)} \medskip}

\newtheorem{theorem}{Theorem}
\newtheorem{lemma}[theorem]{Lemma}
\newtheorem{corollary}[theorem]{Corollary}
\newtheorem{proposition}[theorem]{Proposition}
\newtheorem{conjecture}[theorem]{Conjecture}
\newtheorem{remark}{Remark}
\newtheorem{example}{Example}
\newtheorem{definition}{Definition}
\renewcommand{\textfraction}{0.0}
\newcommand{\dst}{\displaystyle}
\newcommand{\minx}{\mbox{\( \dst \min_{x \in X} \)}}
\newcommand{\Efx}{\mbox{\( \dst E f (x, \xi) \)}}
\newcommand{\Efxhat}{\mbox{\( \dst E f (\hat{x}, \xi) \)}}
\newcommand{\hxx}{\mbox{\( \hat{x} \)}}
\newcommand{\bpi}{\bar{\pi}}
\newcommand{\xx}{\mbox{\( x \)}}
\newcommand{\txxi}{\mbox{\(\xi\)}}
\newcommand{\var}{\mbox{var}}
\newcommand{\cF}{{\cal F}}
\newcommand{\cG}{{\cal G}}
\newcommand{\cN}{{\cal N}}
\newcommand{\cO}{{\cal O}}
\newcommand{\txi}{{\xi}}
\newcommand{\PP}{\mbox{\(SP\)}}
\newcommand{\PPn}{\mbox{\(SP_n\)}}
\newcommand{\PPnx}{\mbox{\(SP_{n_x}\)}}
\newcommand{\noi}{\noindent}
\renewcommand{\ss}{\smallskip}
\newcommand{\ms}{\medskip}
\newcommand{\bs}{\bigskip}
\newcommand{\st}{\mbox{s.t.}}
\newcommand{\wpo}{\mbox{wp1}}
\newcommand{\iid}{\mbox{i.i.d.\ }}
\newcommand{\vsmo}{\vspace*{-0.1in}}
\newcommand{\vsmt}{\vspace*{-0.2in}}
\newcommand{\vso}{\vspace*{0.1in}}
\newcommand{\vst}{\vspace*{0.2in}}
\newcommand{\mc}{\multicolumn}
\newcommand{\cP}{{\cal P}}
\newcommand{\underv}{\mbox{$\underbar{$v$}$}}
%\allowdisplaybreaks 

\renewcommand{\P}{{\mathbb P}}
\newcommand{\E}{{\mathbb E}}
\newcommand{\R}{{\mathbb R}}
\renewcommand{\Re}{{\mathbb R}}
\newcommand{\mbf}{\mathbf}

\bibliographystyle{plain}

\begin{document}
%0.27
\baselineskip0.25in

\begin{center}
\begin{large}
\begin{bf}

Optimizing Crashing Decisions in Project Management Problem with Disruptions \ms

\today \ms
\end{bf}
\end{large}
\end{center}

\section{Introduction} \label{sec:intro}
	\begin{itemize}
		\item Literature review: PERT set up, applications.
		\item Literature review: stochastic disruptions.
		\item Paper structure.
	\end{itemize}
	\textcolor{blue}{One sentence background/history on project management.} A project could be viewed as a collection of activities, each of which will consume some time and resources. There will be precedence relationships between activities due to logical or technological considerations. The objective is to find the smallest amount of time needed to finish all activities. The project could be represented as an activity network where the length of each link shows the duration of an activity, and the direction of it shows the precedence relationship. An activity cannot start until all its predecessors are completed. Two dummy nodes \((S,T)\) are created in the network to represent the start and the end of the project. In the network setting with no uncertainty, finding the shortest project span is equivalent to finding the longest path from \(S\) to \(T\). Since the activity network is usually acyclic, it is easy to use shortest path algorithms to find a longest path in polynomial time. Multiple projects can be processed at the same time without a upper limit, as long as the precedence requirement is satisfied. More details about the activity network are discussed in \cite{Elmaghraby77}.\\
	\newline 
	In the planning stage of a project, decisions can be made to crash a certain set of activities in order to achieve the shortest project span. Here crashing an activity means accelerating its progress or compress its duration. In this paper, a discrete set of crashing options will be given to the decision makers, each with a fractional number which equals to the crashed duration divided by the original duration. One project can only be crashed once with one specific option. Each option will incur a certain cost as well and the total cost of crashing could not exceed the budget. Static crashing optimization problem is first researched in \cite{fulkerson1961network, kelley1961criticalpath}. If we assume a finite set of crashing options, the problem could be modeled as a mixed integer program. \textcolor{blue}{list the original crashing problem literature} The static model can be extended to incorporate uncertainty of activity durations. Monte Carlo simulation methods are used to estimate the expected project span given distributions of activity lengths, since it is really possible to provide an analytical expression of the true expected project span. \textcolor{blue}{list the literature of stochastic crashing optimization problems.} Another approach to handle the uncertainty is the robust optimization method. In the robust optimization setting, the objective is to minimize the worst case project span within the uncertainty set. \textcolor{blue}{list the literature of robust crashing optimization problems}\\
	\newline
	The uncertainty in our paper lies in the timing and the magnitude of the stochastic disruption, which is modeled in a different way from the random variable distribution in the stochastic programming setting, or from the uncertainty set in the robust optimization setting. A stochastic disruption is an event that may occur any time in the time horizon and change the system parameters significantly. Yu and Qi \cite{yu2004disruptionmgt} introduced scenario-based optimization models and applied it to solve airline scheduling problems. Morton et al. \cite{morton2009sealift} introduced the modeling of a sealift scheduling problem under finite number of stochastic disruptions within a stochastic programming structure. This structure ``falls between standard two-stage and multi-stage stochastic programs for a multi-period problem" and reduces the size of the problem to a quadratic growth in the number of time periods. Our setting inherits the philosophy of \cite{morton2009sealift} but enhances the model by allowing continuous disruption time instead of the fixed stages. \textcolor{blue}{list the past papers about the (lacked) work in the stochastic disruption}.\\
	\newline
	Since it is a new model of uncertainty, we first formally describe the crashing optimization problem with disruptions. Given the limited number of disruption occurrence, the problem can be formulated as a stochastic mixed integer program and we will present the extensive formulation in Section~\ref{sec:formulation}. The large scale and the discrete non-convex nature of the extended formulation means it may not be solved efficiently to desirable tolerance level. In Section~\ref{sec:decomposition}, a branch-and-bound method based on Benders decomposition is developed to solve the crashing optimization problem with a disruption. We show such a decomposition method will solve the integer program to the exactness within finite number of iterations. \\
	\newline
	The experiment results are presented in Section~\ref{sec:results}, including the comparison between the quality of our solution and the static solution obtained by solving the crashing optimization, and the time improvement by using the decomposition method in Section~\ref{sec:decomposition} compared to solving the extensive formulation. The experiment result shows \textcolor{blue}{append summary of the experiment results here.} We will conclude our paper with remarks on potential extensions of this model in Section~\ref{sec:conclusions}.
	
\section{Problem Formulation} \label{sec:formulation}
	\begin{itemize}
		\item Static PERT problem formulation/ Stochastic PERT literature review: simulation optimization/ Distributionally robust: Natarajan
		\item Assumptions: not affected by the disruption once started.
	\end{itemize}
	The deterministic project crashing optimization problem is widely researched since 1960s \cite{fulkerson1961network, kelley1961criticalpath}. Suppose the activity network is represented by a directed graph \(\mathcal{G} = (I,\mathcal{A})\), where the set of activities is denoted by \(I\) and their precedence relationship is characterized by \(\mathcal{A}\). The budget limit \(B\) and the nominal length of each activity \(D_i\) are given. For each activity \(i \in I\), there is a finite set of crashing options \(j \in J_i\) and each of them will incur a cost of \(b_{ij}\) while cutting down the nominal activity length to \(1 - e_{ij}\) of the nominal length.  We assume each activity could be crashed at most once and the problem could be formulated as follows. 
	\begin{subequations} \label{prob:static}
		\begin{align}
			\min \quad & t_N &\\
			\text{s.t.} \quad & t_N \geq t_i & \forall i \in I \\
			& t_k - t_i \geq D_{i}(1 - \sum_{j \in J_i} x_{ij} e_{ij}) & \forall i \in I, (i,k) \in \mathcal{A}\\
			& \sum_{i \in I} \sum_{j \in J_i} b_{ij}x_{ij} \leq B & \\
			& \sum_{j \in J_i} x_{ij} \leq 1 & \forall i \in I\\
			& t_i \geq 0 & \forall i \in I\\
			& x_{ij} \in \{0,1\} & \forall i \in I, j \in J_i&
		\end{align}
	\end{subequations}
	Based on this formulation, we can derive the stochastic optimization and the robust optimization models with reference to \textcolor{blue}{list the literature of stochastic/robust crashing optimization problems}. We omit these formulations in this paper since our uncertainty model is different from the literature listed above.\\
	\newline
	We assume at most one stochastic disruption will occur in the project span. The disruption will not affect the activities that have already started but will change the length of activities starting after it according to some given finite discrete distributions. The decision of crashing has to be made the same time when the activity starts and cannot be changed after that. Since there is at most one disruption, we can model the problem as a two-stage stochastic mixed integer program. Notice that the definition of the first stage is different from the traditional stochastic program setting. Here the first stage contains all the decisions if no disruption occurs and the second stage will characterize the decisions for each realization of the disruption. This means that the first stage decision variables are not limited to a certain time period but can be all over the entire time horizon, which allows us to model continuous disruption time.\\
	\newline
	The notation for the model is displayed as follows:
	\begin{table}[H]
		\begin{tabular}{ l l l l }
			\multicolumn{4}{l}{Indices and index sets} \\
			\\
			\(I\) & \(\qquad\) & the set of activities;&\\
			\(J_i\) & \(\qquad\) & the set of crashing options for activity \(i\), \(i \in I\);&\\
			\(\Omega\) & \(\qquad\) & the set of possible realizations of disruption;&\\
			\(\mathcal{A}\) &\(\qquad\) & set of arcs which represents the precedence relationship;&\\
			\\
			\multicolumn{4}{l}{Parameters} \\
			\\
			\(D_{i}\)& \(\qquad\) & original duration of activity \(i\), \(i \in I\);&\\
			\(e_{ij}\) & \(\qquad\) & effectiveness of crashing option \(j\), \(i \in I, j \in J_i\);&\\
			\(B\) & \(\qquad\) & total budget;&\\
			\(b_{ij}\) & \(\qquad\) & cost of crashing option \(j\), \(i \in I, j \in J_i\);&\\
			\(H^\omega\) &\(\qquad\) & disruption time in scenario \(\omega\), \(\omega \in \Omega\);&\\
			\(d_{i}^\omega\) & \(\qquad\)&the change of duration of activity \(i\) in realization \(\omega\), &\\
			& \(\qquad\) & if the activity is started after the disruption, \(i \in I, \omega \in \Omega\);& \\
			\(p^\omega\) & \(\qquad\) & the probability of scenario \(\omega\), \(\omega \in \Omega\);& \\
			\(p^0\) & \(\qquad\) & the probability of no disruption;& \\
			\\
			\multicolumn{4}{l}{Decision Variables}\\
			\\
			\(t_{i}\) & \(\qquad\) & nominal starting time of activity \(i\), \(i \in I\);&\\
			\(x_{ij}\) & \(\qquad\) & indicator whether activity \(i\) is crashed by option \(j\) in the nominal plan, \(i \in I, j \in J_i\); &\\
			\(t_{i}^\omega\) & \(\qquad\) & starting time of activity \(i\) under scenario \(\omega\), \(i \in I, \omega \in \Omega\);&\\
			\(x_{ij}^\omega\) & \(\qquad\) & indicator whether activity \(i\) is crashed by option \(j\) under scenario \(\omega\), \(i \in I, j \in J_i, \omega \in \Omega \); &\\
			\(F_i^\omega\) & \(\qquad\) & indicator whether activity \(i\) starts before disruption in realization \(\omega\), \(i \in I, \omega \in \Omega\);&\\
			\(G_i^\omega\) & \(\qquad\) & indicator whether activity \(i\) starts after disruption in realization \(\omega\), \(i \in I, \omega \in \Omega\);&\\
			\(S_{ij}^\omega\) & \(\qquad\) & binary term to linearize the bilinear term \(G_i^\omega x_{ij}^\omega\), \(i \in I, j \in J_{i}, \omega \in \Omega\).&\\
		\end{tabular}
	\end{table}
	\noi The extensive formulation of the two-stage stochastic program is shown as Formulation~(\ref{prob:extensive}). Here \(M\) is a large number to enforce the logic relationships. In this problem, we are minimizing the expected project span by taking the sum of each scenario's project span weighted by the scenario probability. Variables \(F^\omega_i\) and \(G^\omega_i\) are used to identify whether activity \(i\) starts before or after the disruption time (in constraint (\ref{cons:f}) - (\ref{cons:fg})). This is important in our problem setting because the duration of each activity will depend on its temporal relationship to the disruption time, which is reflected in constraint (\ref{cons:duration}). For this duration constraint, the original expression of the duration of activity \(i\) is \((D_i + d_i^\omega G_i^\omega)(1 - \sum_{j \in J_i} e_{ij}x_{ij}^\omega)\). This contains a bilinear term \(G_i^\omega x_{ij}^\omega\), which can be linearized by introducing a binary variable \(S_{ij}^\omega\) and bounding it by constraints (\ref{cons:linearize1}) - (\ref{cons:linearize3}).\\
	\newline
	We have to use a set of non-anticipativity constraints (\ref{cons:tf1}) - (\ref{cons:xf2}), different from the one used in a traditional stochastic programming setting, since the first stage decision variables here spans the entire time horizon. These logic constraints ensure the decisions made before the disruption time in each scenario should be the same as the nominal decisions before the same time point, because decisions made with the same amount information should be the same.\\
	\begin{subequations} \label{prob:extensive}
		\begin{align}
			\min \quad & \sum_{\omega \in \Omega} p^\omega t_N^\omega + p^0 t_N& \\
			\text{s.t.} \quad & t_N^\omega \geq t_i^\omega & \forall i \in I, \omega \in \Omega \\
			& t_N \geq t_i & \\
			& t_i^\omega \geq 0 & \forall i \in I, \omega \in \Omega\\
			& t_i \geq 0 & \forall i \in I\\
			& H^\omega - F_i^\omega M \leq t_i & \forall i \in I, \omega \in \Omega \label{cons:f}\\
			& H^\omega + G_i^\omega M \geq t_i & \forall i \in I, \omega \in \Omega \label{cons:g}\\
			& F_i^\omega + G_i^\omega = 1 & \forall i \in I, \omega \in \Omega \label{cons:fg}\\
			& t_i^\omega + (1 - F_i^\omega)M \geq t_i & \forall i \in I, \omega \in \Omega \label{cons:tf1}\\
			& t_i^\omega - (1 - F_i^\omega)M \leq t_i & \forall i \in I, \omega \in \Omega \label{cons:tf2}\\
			& x_{ij}^\omega + (1 - F_i^\omega)M \geq x_{ij} & \forall i \in I, j \in J_i, \omega \in \Omega \label{cons:xf1}\\
			& x_{ij}^\omega - (1 - F_i^\omega)M \leq x_{ij} & \forall i \in I, j \in J_i, \omega \in \Omega \label{cons:xf2}\\
			& t_k^\omega - t_i^\omega \geq D_i + d_i^\omega G_i^\omega -\sum_{j \in J_i} D_i e_{ij} x_{ij}^\omega - \sum_{j \in J_i} d_i^\omega e_{ij} S_{ij}^\omega & \forall i \in I, j \in J_i, \omega \in \Omega \label{cons:duration}\\
			& \sum_{j \in J_i} x_{ij}^\omega \leq 1 & \forall i \in I, \omega \in \Omega \label{cons:crashLim}\\
			& \sum_{i \in I}\sum_{j \in J_i} b_jx_{ij}^\omega \leq B & \forall \omega \in \Omega \label{cons:budget}\\
			& S_{ij}^\omega \leq G_i^\omega & \forall i \in I, j \in J_i, \omega \in \Omega \label{cons:linearize1}\\
			& S_{ij}^\omega \leq x_{ij}^\omega & \forall i \in I, j \in J_i, \omega \in \Omega \label{cons:linearize2}\\
			& S_{ij}^\omega \geq G_i^\omega + x_{ij}^\omega - 1 & \forall i \in I, j \in J_i, \omega \in \Omega \label{cons:linearize3}\\
			& x_{ij}^\omega \in \{0,1\} & \forall i \in I, j \in J_i, \omega \in \Omega\\
			& F_i^\omega, G_i^\omega \in \{0,1\} & \forall i \in I, \omega \in \Omega\\
			& t_{i}^\omega \geq 0 & \forall i \in I, j \in J_i, \omega \in \Omega
		\end{align}
	\end{subequations}
	
\section{Decomposition Method} \label{sec:decomposition}
	\begin{itemize}
		\item Formulation: master, sub
		\item Cut generation
		\item Branch and bound method
	\end{itemize}
	Problem (\ref{prob:extensive}) is a two-sage stochastic mixed integer program with both binary (\(x\)) and continuous (\(t\)) state variables. A state variable here is a variable whose value is passed from the first stage to the second stage. Two-stage stochastic linear program could be solved nicely using Benders decomposition (\textcolor{blue}{cite Benders decomposition}). However, it is difficult to solve a stochastic mixed integer program with integer variables in the recourse problem because the linear cuts generated in the regular Benders decomposition could not characterize the recourse function exactly. \textcolor{blue}{cite literature about the stochastic integer programs}. \\
	\newline 
	Recently Zou et al. \cite{zou2016nested} propose to generate Lagrangian cuts in the place of Benders cuts to solve multi-stage stochastic programs with only binary state variables. Their method keeps the structure of Benders decomposition and proves the tightness of Lagrangian cuts. Although this result could not be directly applied in our problem due to the continuous state variable \(t\), it provides a pivotal theoretical basis for our decomposition method.\\
	\newline
	First we could separate the extensive formulation (\ref{prob:extensive}) into a master problem and a series of sub-problems as (\ref{prob:master}) and (\ref{prob:sub}). Notice that it is difficult to directly solve the master problem because the function \(f^\omega\) is non-convex and it is hard to derive its analytical form.
	\begin{subequations}
		\label{prob:master}
		\begin{align}
		(M) \quad \min \quad & p^0 t_N + \sum_{\omega \in \Omega} p^\omega f^\omega(x,t) & \\
		\text{s.t.} \quad & t_N \geq t_i & \forall i \in I\\
		& t_k - t_i \geq D_{i}(1 - \sum_{j \in J_i} x_{ij} e_{ij}) & \forall i \in I, (i,k) \in \mathcal{A}\\
		& \sum_{i \in I} \sum_{j \in J_i} b_{ij}x_{ij} \leq B & \\
		& \sum_{j \in J_i} x_{ij} \leq 1 & \forall i \in I\\
		& t_i \geq 0 & \forall i \in I\\
		& x_{ij} \in \{0,1\}. & \forall i \in I, j \in J_i&
		\end{align}
	\end{subequations}
	\begin{subequations}
		\label{prob:sub}
		\begin{align}
			(S^\omega) \quad f^\omega(\hat{x},\hat{t}) = \min \quad & t_N & \\
			\text{s.t.} \quad & t_N \geq t_i & \forall i \in I \\
			& H^\omega - F_i M \leq t_i & \forall i \in I \\
			& H^\omega + G_i M \geq t_i & \forall i \in I \\
			& F_i + G_i = 1 & \forall i \in I \\
			& t_i + (1-F_i) M \geq \hat{t}_i & \forall i \in I\\
			& t_i - (1-F_i) M \leq \hat{t}_i & \forall i \in I \\
			& x_{ij} + (1-F_i) \geq \hat{x}_{ij} & \forall i \in I, j \in J_i \\
			& x_{ij} - (1-F_i) \leq \hat{x}_{ij} & \forall i \in I, j \in J_i \\
			& t_k - t_i \geq D_i + d_i^\omega G_i -\sum_{j \in J_i} D_i e_{ij} x_{ij} - \sum_{j \in J_i} d_i^\omega e_{ij} S_{ij} & \forall i \in I, (i,k) \in \mathcal{A}\\
			&\sum_{i \in I} \sum_{j \in J_i} b_{ij} x_{ij} \leq B &\\
			& \sum_{j \in J_i} x_{ij} \leq 1 & \forall i \in I\\
			& S_{ij} \leq G_i & \forall i \in I, j \in J_i \\
			& S_{ij} \leq x_{ij} & \forall i \in I, j \in J_i\\
			& S_{ij} \geq G_i + x_{ij} - 1 & \forall i \in I, j \in J_i\\
			& t_i \geq 0 & \forall i \in I \\
			& x_{ij} \in \{0,1\}. & \forall i \in I, j \in J_i
		\end{align}
	\end{subequations}
	\textcolor{blue}{Introduce the Lagrangian cut and how it is generated.}
	Since our state variable contains both continuous variables \(t\) and binary variables \(x\), the Lagrangian cuts will only be a lower bound but the tightness could not be guaranteed. This means that if we carry out a cutting plane scheme using the Lagrangian cuts, we will not solve the problem to the exactness but will only obtain a lower bound. \\
	\newline
	One important observation to help with this issue is that if we can determine the temporal relationship between every activity and the disruption, the restricted version of the function \(f^\omega\) will be convex in \(t\), and we could find a tight and valid Lagrangian cut similar to those in \cite{zou2016nested}. This means, for example, if for a specific scenario \(\omega\), \(H^\omega = 5\), and for each activity \(i \in I\) we know whether its starting time \(t_i\) is less than, larger than or equal to \(H^\omega = 5\), then there is a finite set of linear constraints that could be generated to fully characterize \(f^\omega(x,t)\) in formulation~(\ref{prob:master}). This property is true mainly because once we know the temporal relationship between the starting time of each activity and the disruption time, it is always optimal to start the activity at its earliest possible time. We will prove this property as follows. \\
	\newline
	\textcolor{blue}{State the Proofs of the tightness/validity of the Lagrangian cuts}\\
	\newline
	The temporal relationship could be determined by a branch-and-bound scheme. Although theoretically for each scenario there are exponential number of possible temporal relationships because each activity can either be before the disruption time or after that, there are only a small number of activities, of which the optimal starting time of the deterministic problem (\ref{prob:static}) is close to the disruption time, that really matter. If an activity should be starting much earlier than the disruption time in the deterministic problem, it usually will significantly increase the length of the project span by delaying the start of such activity. If an activity cannot start until a time much later than the disruption time, it is usually not possible to make it start before the disruption time without committing crashing resources heavily on its precedent activities, which may likely lead to an inferior solution.

\section{Computational Results} \label{sec:results}

\section{Conclusions} \label{sec:conclusions}
	\begin{itemize}
		\item Contributions
		\item Further directions
	\end{itemize}

\bibliographystyle{plain}
\bibliography{PERT_Bib}

\end{document}