\documentclass[11pt]{article}
\usepackage[small]{titlesec}
\usepackage[top = 0.66in,textwidth = 6.5in, textheight=9.1in]{geometry}

\usepackage{amsmath}
\usepackage{graphicx}
\usepackage{latexsym}
\usepackage{color}
\usepackage{amssymb}
\usepackage{tabularx}
\usepackage{fancyhdr}
\usepackage{verbatim}
\usepackage{multirow}
\usepackage{framed}
\usepackage{natbib}
\usepackage{float, subfig}
\usepackage{enumitem}
\usepackage{mathtools}
\usepackage{mathrsfs}
\usepackage{amsfonts}
\usepackage{listings}
\usepackage{amsthm}
\usepackage{grffile}
\usepackage{sidecap}
\usepackage{pbox}
\usepackage{algorithm}
\usepackage[noend]{algpseudocode}

\def\qed{\hfill{\(\vcenter{\hrule height1pt \hbox{\vrule width1pt height5pt
     \kern5pt \vrule width1pt} \hrule height1pt}\)} \medskip}

\newtheorem{theorem}{Theorem}
\newtheorem{lemma}[theorem]{Lemma}
\newtheorem{corollary}[theorem]{Corollary}
\newtheorem{proposition}[theorem]{Proposition}
\newtheorem{conjecture}[theorem]{Conjecture}
\newtheorem{remark}{Remark}
\newtheorem{example}{Example}
\newtheorem{definition}{Definition}
\renewcommand{\textfraction}{0.0}
\newcommand{\dst}{\displaystyle}
\newcommand{\minx}{\mbox{\( \dst \min_{x \in X} \)}}
\newcommand{\Efx}{\mbox{\( \dst E f (x, \xi) \)}}
\newcommand{\Efxhat}{\mbox{\( \dst E f (\hat{x}, \xi) \)}}
\newcommand{\hxx}{\mbox{\( \hat{x} \)}}
\newcommand{\bpi}{\bar{\pi}}
\newcommand{\xx}{\mbox{\( x \)}}
\newcommand{\txxi}{\mbox{\(\xi\)}}
\newcommand{\var}{\mbox{var}}
\newcommand{\cF}{{\cal F}}
\newcommand{\cG}{{\cal G}}
\newcommand{\cN}{{\cal N}}
\newcommand{\cO}{{\cal O}}
\newcommand{\txi}{{\xi}}
\newcommand{\PP}{\mbox{\(SP\)}}
\newcommand{\PPn}{\mbox{\(SP_n\)}}
\newcommand{\PPnx}{\mbox{\(SP_{n_x}\)}}
\newcommand{\noi}{\noindent}
\renewcommand{\ss}{\smallskip}
\newcommand{\ms}{\medskip}
\newcommand{\bs}{\bigskip}
\newcommand{\st}{\mbox{s.t.}}
\newcommand{\wpo}{\mbox{wp1}}
\newcommand{\iid}{\mbox{i.i.d.\ }}
\newcommand{\vsmo}{\vspace*{-0.1in}}
\newcommand{\vsmt}{\vspace*{-0.2in}}
\newcommand{\vso}{\vspace*{0.1in}}
\newcommand{\vst}{\vspace*{0.2in}}
\newcommand{\mc}{\multicolumn}
\newcommand{\cP}{{\cal P}}
\newcommand{\underv}{\mbox{$\underbar{$v$}$}}
\allowdisplaybreaks 

\renewcommand{\P}{{\mathbb P}}
\newcommand{\E}{{\mathbb E}}
\newcommand{\R}{{\mathbb R}}
\renewcommand{\Re}{{\mathbb R}}
\newcommand{\mbf}{\mathbf}

\bibliographystyle{plain}

\begin{document}
%0.27
\baselineskip0.25in

\begin{center}
\begin{large}
\begin{bf}

Optimizing Crashing Decisions in Project Management Problem with Disruptions \ms

\today \ms
\end{bf}
\end{large}
\end{center}

\section{Introduction} \label{sec:intro}
	Project management has been a topic studied by many operations researchers using ``engineering science and optimization theory" \cite{soderlund2004building}. In order to derive an abstract model, a project could be viewed as a collection of activities, each of which will consume some time and resources. There will be precedence relationships between activities due to logical or technological considerations. The objective is to find the smallest amount of time needed to finish all activities. The project could be represented as an activity network where the length of each link shows the duration of an activity, and the direction of it shows the precedence relationship. An activity cannot start until all its predecessors are completed. Two dummy nodes \((S,T)\) are created in the network to represent the start and the end of the project. In the network setting with no uncertainty, finding the shortest project span is equivalent to finding the longest path from \(S\) to \(T\). Since the activity network is usually acyclic, it is easy to use shortest path algorithms to find a longest path in polynomial time. Multiple projects can be processed at the same time without a upper limit, as long as the precedence requirement is satisfied. More details about the activity network are discussed in \cite{Elmaghraby77}.\\
	\newline 
	In the planning stage of a project, decisions can be made to crash a certain set of activities in order to achieve the shortest project span. Here crashing an activity means accelerating its progress or compress its duration. In this paper, a discrete set of crashing options will be given to the decision makers, each with a fractional number which equals to the crashed duration divided by the original duration. One project can only be crashed once with one specific option. Each option will incur a certain cost as well and the total cost of crashing could not exceed the budget. Static crashing optimization problem is first researched in \cite{fulkerson1961network, kelley1961criticalpath}. If we assume a finite set of crashing options, the problem could be modeled as a mixed integer program. The static model can be extended to incorporate uncertainty of activity durations. Monte Carlo simulation methods are used to estimate the expected project span given distributions of activity lengths, since it is not really possible to provide an analytical expression of the true expected project span \cite{burt1971conditional,van1963letter}. Heuristics and simulation-based algorithms have been developed to solve the stochastic project crashing problem \cite{aghaie2009ant,ke2014genetic,kim2007heuristic}. Another approach to handle the uncertainty is the robust optimization method. In the robust optimization setting, the objective is to minimize the worst case project span within the uncertainty set. \textcolor{blue}{select a few papers about robust project crashing and describe what they did.}\\
	\newline
	The uncertainty in our paper lies in the timing and the magnitude of the stochastic disruption, which is modeled in a different way from the random variable distribution in the stochastic programming setting, or from the uncertainty set in the robust optimization setting. A stochastic disruption is an event that may occur any time in the time horizon and change the system parameters significantly. There have been a few papers in applying this idea when the disruption could only occur in a set of specified time periods. Yu and Qi \cite{yu2004disruptionmgt} introduced scenario-based optimization models and applied it to solve airline scheduling problems. Morton et al. \cite{morton2009sealift} introduced the modeling of a sealift scheduling problem under finite number of stochastic disruptions within a stochastic programming structure. This structure ``falls between standard two-stage and multi-stage stochastic programs for a multi-period problem" and reduces the size of the problem to a quadratic growth in the number of time periods. Our setting inherits the philosophy of \cite{morton2009sealift} but enhances the model by allowing continuous disruption time instead of fixed time periods.\\
	\newline
	We first formally describe the crashing optimization problem under stochastic disruptions. Given the limited number of disruption occurrence, the problem can be formulated as a stochastic mixed integer program and we will present the extensive formulation in Section~\ref{sec:formulation}. The large scale and the discrete non-convex nature of the extended formulation means it may not be solved efficiently to desirable tolerance level. In Section~\ref{sec:decomposition}, a branch-and-cut method based on Benders decomposition is developed to solve the crashing optimization problem with a disruption. We show such a decomposition method will solve the integer program to the exactness within finite number of iterations.\\
	\newline
	The experiment results are presented in Section~\ref{sec:results}, including the comparison between the quality of our solution and the solution obtained by solving a  crashing optimization problem with a deterministic disruption, and the computational performance of the decomposition method in Section~\ref{sec:decomposition}, compared to that of solving the extensive formulation. The experiment result shows \textcolor{blue}{(append summary of the experiment results here.)} We will conclude our paper with remarks on potential extensions of this model in Section~\ref{sec:conclusions}.
	
\section{Problem Formulation} \label{sec:formulation}
	The deterministic project crashing optimization problem is widely researched since 1960s \cite{fulkerson1961network, kelley1961criticalpath}. Suppose the activity network is represented by a directed graph \(\mathcal{G} = (I,\mathcal{A})\), where the set of activities is denoted by \(I\) and their precedence relationship is characterized by \(\mathcal{A}\). Each element in set \(\mathcal{A}\) is a two dimensional vector \((i,k)\), which indicates that we have to finish activity \(i\) before starting activity \(k\), \(i,k \in I\).  The nominal length of each activity \(D_i\) is given, which means if there is no disruption, activity \(i\) will require \(D_i\) amount of time to finish. For each activity \(i \in I\), there is a finite set of crashing options \(j \in J_i\) and each of them will incur a cost of \(b_{ij}\). Applying crashing option \(j\) will decrease the activity length to \(1 - e_{ij}\) of its non-crashed length. We assume each activity can be crashed at most once and the total crashing cost could not be over a given budget \(B\). The deterministic project crashing problem could be formulated as follows:
	\begin{subequations} \label{prob:static}
		\begin{align}
			\min \quad & t_N &\\
			\text{s.t.} \quad & t_N \geq t_i & \forall i \in I \label{cons:dotN}\\
			& t_k - t_i \geq D_{i}(1 - \sum_{j \in J_i} x_{ij} e_{ij}) & \forall i \in I, (i,k) \in \mathcal{A} \label{cons:dott}\\
			& \sum_{i \in I} \sum_{j \in J_i} b_{ij}x_{ij} \leq B & \label{cons:doBudget}\\
			& \sum_{j \in J_i} x_{ij} \leq 1 & \forall i \in I\\
			& t_i \geq 0 & \forall i \in I\\
			& x_{ij} \in \{0,1\} & \forall i \in I, j \in J_i&
		\end{align}
	\end{subequations}
	Based on this formulation, we can derive the stochastic optimization and the robust optimization models with reference to \textcolor{blue}{list the literature of stochastic/robust crashing optimization problems}. We omit these formulations in this paper since our uncertainty model is different from the literature listed above.\\
	\newline
	We assume at most one stochastic disruption will occur in the project span. The disruption will not affect the activities that have already started but will change the length of activities starting after it according to some given finite discrete distributions. The decision of crashing has to be made the same time when the activity starts and cannot be changed after that. Since there is at most one disruption, we can model the problem as a two-stage stochastic mixed integer program. Notice that the definition of the first stage is different from the traditional stochastic program setting. Here the first stage contains all the decisions if no disruption occurs and the second stage will characterize the decisions for each realization of the disruption. This means that the first stage decision variables are not limited to a certain time period but can be all over the entire time horizon, which allows us to model continuous disruption time.\\
	\newline
	The notation for the model is displayed as follows:
	\begin{table}[H]
		\begin{tabular}{ l l l l }
			\multicolumn{4}{l}{Indices and index sets} \\
			\\
			\(I\) & \(\qquad\) & the set of activities;&\\
			\(I_i\) & \(\qquad\) & the set of activities which directly lead to activity \(i\); &\\
			\(J_i\) & \(\qquad\) & the set of crashing options for activity \(i\), \(i \in I\);&\\
			\(\Omega\) & \(\qquad\) & the set of possible realizations of disruption;&\\
			\(\mathcal{A}\) &\(\qquad\) & set of arcs which represents the precedence relationship;&\\
			\\
			\multicolumn{4}{l}{Parameters} \\
			\\
			\(D_{i}\)& \(\qquad\) & original duration of activity \(i\), \(i \in I\);&\\
			\(e_{ij}\) & \(\qquad\) & effectiveness of crashing option \(j\), \(i \in I, j \in J_i\);&\\
			\(B\) & \(\qquad\) & total budget;&\\
			\(b_{ij}\) & \(\qquad\) & cost of crashing option \(j\), \(i \in I, j \in J_i\);&\\
			\(H^\omega\) &\(\qquad\) & disruption time in scenario \(\omega\), \(\omega \in \Omega\);&\\
			\(d_{i}^\omega\) & \(\qquad\)&the change of duration of activity \(i\) in realization \(\omega\), &\\
			& \(\qquad\) & if the activity is started after the disruption, \(i \in I, \omega \in \Omega\);& \\
			\(p^\omega\) & \(\qquad\) & the probability of scenario \(\omega\), \(\omega \in \Omega\);& \\
			\(p^0\) & \(\qquad\) & the probability of no disruption;& \\
			\\
			\multicolumn{4}{l}{Decision Variables}\\
			\\
			\(t_{i}\) & \(\qquad\) & nominal starting time of activity \(i\), \(i \in I\);&\\
			\(x_{ij}\) & \(\qquad\) & indicator whether activity \(i\) is crashed by option \(j\) in the nominal plan, \(i \in I, j \in J_i\); &\\
			\(t_{i}^\omega\) & \(\qquad\) & starting time of activity \(i\) under scenario \(\omega\), \(i \in I, \omega \in \Omega\);&\\
			\(x_{ij}^\omega\) & \(\qquad\) & indicator whether activity \(i\) is crashed by option \(j\) under scenario \(\omega\), \(i \in I, j \in J_i, \omega \in \Omega \); &\\
			\(F_i^\omega\) & \(\qquad\) & indicator whether activity \(i\) starts before disruption in realization \(\omega\), \(i \in I, \omega \in \Omega\);&\\
			\(G_i^\omega\) & \(\qquad\) & indicator whether activity \(i\) starts after disruption in realization \(\omega\), \(i \in I, \omega \in \Omega\);&\\
			\(S_{ij}^\omega\) & \(\qquad\) & binary term to linearize the bilinear term \(G_i^\omega x_{ij}^\omega\), \(i \in I, j \in J_{i}, \omega \in \Omega\).&\\
		\end{tabular}
	\end{table}
	\noi The extensive formulation of the two-stage stochastic program is shown as Formulation~(\ref{prob:extensive}). Here \(M\) is a large number to enforce the logic relationships. In this problem, we are minimizing the expected project span by taking the sum of each scenario's project span weighted by the scenario probability. Variables \(F^\omega_i\) and \(G^\omega_i\) are used to identify whether activity \(i\) starts before or after the disruption time (in constraint (\ref{cons:F}) - (\ref{cons:FG})). This is important in our problem setting because the duration of each activity will depend on its temporal relationship to the disruption time, which is reflected in constraint (\ref{cons:duration}). For this duration constraint, the original expression of the duration of activity \(i\) is \((D_i + d_i^\omega G_i^\omega)(1 - \sum_{j \in J_i} e_{ij}x_{ij}^\omega)\). This contains a bilinear term \(G_i^\omega x_{ij}^\omega\), which can be linearized by introducing a binary variable \(S_{ij}^\omega\) and bounding it by constraints (\ref{cons:linearize1}) - (\ref{cons:linearize3}).\\
	\newline
	We have to use a set of non-anticipativity constraints (\ref{cons:tf1}) - (\ref{cons:xf2}), different from the one used in a traditional stochastic programming setting, since the first stage decision variables here spans the entire time horizon. These logic constraints ensure the decisions made before the disruption time in each scenario should be the same as the nominal decisions before the same time point, because decisions made with the same amount information should be the same.\\
	\begin{subequations} \label{prob:extensive}
		\begin{align}
			\min \quad & \sum_{\omega \in \Omega} p^\omega t_N^\omega + p^0 t_N& \\
			\text{s.t.} \quad & t_N^\omega \geq t_i^\omega & \forall i \in I, \omega \in \Omega \\
			& t_N \geq t_i & \label{cons:tN}\\
			& t_i^\omega \geq 0 & \forall i \in I, \omega \in \Omega\\
			& t_i \geq 0 & \forall i \in I\\
			& H^\omega - F_i^\omega M \leq t_i & \forall i \in I, \omega \in \Omega \label{cons:F}\\
			& H^\omega + G_i^\omega M \geq t_i & \forall i \in I, \omega \in \Omega \label{cons:G}\\
			& F_i^\omega + G_i^\omega = 1 & \forall i \in I, \omega \in \Omega \label{cons:FG}\\
			& t_i^\omega + (1 - F_i^\omega)M \geq t_i & \forall i \in I, \omega \in \Omega \label{cons:tF1}\\
			& t_i^\omega - (1 - F_i^\omega)M \leq t_i & \forall i \in I, \omega \in \Omega \label{cons:tF2}\\
			& x_{ij}^\omega + (1 - F_i^\omega)M \geq x_{ij} & \forall i \in I, j \in J_i, \omega \in \Omega \label{cons:xF1}\\
			& x_{ij}^\omega - (1 - F_i^\omega)M \leq x_{ij} & \forall i \in I, j \in J_i, \omega \in \Omega \label{cons:xF2}\\
			& t_k^\omega - t_i^\omega \geq D_i + d_i^\omega G_i^\omega -\sum_{j \in J_i} D_i e_{ij} x_{ij}^\omega - \sum_{j \in J_i} d_i^\omega e_{ij} S_{ij}^\omega & \forall i \in I, j \in J_i, \omega \in \Omega \label{cons:duration}\\
			& \sum_{j \in J_i} x_{ij}^\omega \leq 1 & \forall i \in I, \omega \in \Omega \label{cons:crashLim}\\
			& \sum_{i \in I}\sum_{j \in J_i} b_jx_{ij}^\omega \leq B & \forall \omega \in \Omega \label{cons:budget}\\
			& S_{ij}^\omega \leq G_i^\omega & \forall i \in I, j \in J_i, \omega \in \Omega \label{cons:linearize1}\\
			& S_{ij}^\omega \leq x_{ij}^\omega & \forall i \in I, j \in J_i, \omega \in \Omega \label{cons:linearize2}\\
			& S_{ij}^\omega \geq G_i^\omega + x_{ij}^\omega - 1 & \forall i \in I, j \in J_i, \omega \in \Omega \label{cons:linearize3}\\
			& x_{ij}^\omega \in \{0,1\} & \forall i \in I, j \in J_i, \omega \in \Omega\\
			& F_i^\omega, G_i^\omega \in \{0,1\}. & \forall i \in I, \omega \in \Omega
		\end{align}
	\end{subequations}
	
\section{Decomposition Method} \label{sec:decomposition}
	\begin{itemize}
		\item Formulation: master, sub
		\item Cut generation
		\item Branch and bound method
	\end{itemize}
	Problem (\ref{prob:extensive}) is a two-sage stochastic mixed integer program with both binary (\(x\)) and continuous (\(t\)) state variables. A state variable here is a variable whose value is passed from the first stage to the second stage. Two-stage stochastic linear program could be solved nicely using Benders decomposition (\textcolor{blue}{cite Benders decomposition}). \\
	\newline 
	For our crashing optimization problem, we could separate the extensive formulation (\ref{prob:extensive}) into a master problem and a series of sub-problems as (\ref{prob:master}) and (\ref{prob:sub}). Constraint (\ref{cons:linearCuts}) are generated by solving the sub-problems iteratively for the optimal multipliers \(\pi\) and \(\lambda\). In a regular multi-cut Benders decomposition case, the sub problem is a convex optimization problem, usually a linear program. This means a finite number of constraints on \(\theta^\omega\) can fully characterize the function \(f^\omega(x,t)\). However, in this case \(f^\omega(x,t)\) is a mixed integer linear program. It is difficult to solve a stochastic mixed integer program with integer variables in the recourse problem because the linear cuts generated in the regular Benders decomposition could not characterize the recourse function exactly. \textcolor{blue}{cite literature about the stochastic integer programs}.
	\begin{subequations}
		\label{prob:master}
		\begin{align}
		(M) \quad \min \quad & p^0 t_N + \sum_{\omega \in \Omega} p^\omega \theta^\omega & \\
		\text{s.t.} \quad & t_N \geq t_i & \forall i \in I\\
		& t_k - t_i \geq D_{i}(1 - \sum_{j \in J_i} x_{ij} e_{ij}) & \forall i \in I, (i,k) \in \mathcal{A}\\
		& \sum_{i \in I} \sum_{j \in J_i} b_{ij}x_{ij} \leq B & \\
		& \sum_{j \in J_i} x_{ij} \leq 1 & \forall i \in I\\
		& \theta^\omega \geq \sum_{i \in I} \pi_i^{\omega,k^\omega} t_i + \sum_{i \in I} \sum_{j \in J_i} \lambda_{ij}^{\omega,k^\omega} x_{ij} + v^{\omega,k^\omega} & \forall \omega \in \Omega, k^\omega = 1,2, \dots \label{cons:linearCuts}\\
		& t_i \geq 0 & \forall i \in I\\
		& x_{ij} \in \{0,1\}. & \forall i \in I, j \in J_i&
		\end{align}
	\end{subequations}
	\begin{subequations}
		\label{prob:sub}
		\begin{align}
			(S^\omega) \quad f^\omega(\hat{x},\hat{t}) = \min \quad & t_N & \\
			\text{s.t.} \quad & t_N \geq t_i & \forall i \in I \\
			& H^\omega - F_i M \leq t_i & \forall i \in I \\
			& H^\omega + G_i M \geq t_i & \forall i \in I \\
			& F_i + G_i = 1 & \forall i \in I \\
			& t_i + (1-F_i) M \geq \hat{t}_i & \forall i \in I\\
			& t_i - (1-F_i) M \leq \hat{t}_i & \forall i \in I \\
			& x_{ij} + (1-F_i) \geq \hat{x}_{ij} & \forall i \in I, j \in J_i \\
			& x_{ij} - (1-F_i) \leq \hat{x}_{ij} & \forall i \in I, j \in J_i \\
			& t_k - t_i \geq D_i + d_i^\omega G_i -\sum_{j \in J_i} D_i e_{ij} x_{ij} - \sum_{j \in J_i} d_i^\omega e_{ij} S_{ij} & \forall i \in I, (i,k) \in \mathcal{A}\\
			&\sum_{i \in I} \sum_{j \in J_i} b_{ij} x_{ij} \leq B &\\
			& \sum_{j \in J_i} x_{ij} \leq 1 & \forall i \in I\\
			& S_{ij} \leq G_i & \forall i \in I, j \in J_i \\
			& S_{ij} \leq x_{ij} & \forall i \in I, j \in J_i\\
			& S_{ij} \geq G_i + x_{ij} - 1 & \forall i \in I, j \in J_i\\
			& t_i \geq 0 & \forall i \in I \\
			& x_{ij} \in \{0,1\}. & \forall i \in I, j \in J_i
		\end{align}
	\end{subequations}
	Recently Zou et al. \cite{zou2016nested} propose to generate Lagrangian cuts in the place of Benders cuts to solve multi-stage stochastic programs with only binary state variables. Their method keeps the structure of Benders decomposition and proves the tightness of Lagrangian cuts. To generate the Lagrangian multipliers, a local copy (\(\tau\) and \(z\)) of the state variable (\(\hat{t}\) and \(\hat{x}\)) is first created as in constraint (\ref{cons:local1}) and (\ref{cons:local2}) of formulation (\ref{prob:subLocal}). We can then relax the constraint (\ref{cons:local1}) and (\ref{cons:local2}) to create a Lagrangian dual problem (\ref{prob:LagDual}). The optimal solution  \(\pi^*\) and \(\lambda^*\) to the Lagrangian dual problem will be used in the Lagrangian cuts as in constraint (\ref{cons:linearCuts}). Problem (\ref{prob:LagDual}) can be solved by a subgradient method \textcolor{blue}{cite Fisher's tutorial for solving the Lagrangian Relaxation problem}.\\
	\begin{subequations}
		\label{prob:subLocal}
		\begin{align}
		(S^\omega_L) \quad \min \quad & t_N & \\
		\text{s.t.} \quad & t_N \geq t_i & \forall i \in I \label{cons:stN}\\
		& H^\omega - F_i M \leq t_i & \forall i \in I \\
		& H^\omega + G_i M \geq t_i & \forall i \in I \\
		& F_i + G_i = 1 & \forall i \in I \\
		& t_i + (1-F_i) M \geq \tau_i & \forall i \in I\\
		& t_i - (1-F_i) M \leq \tau_i & \forall i \in I \\
		& x_{ij} + (1-F_i) \geq z_{ij} & \forall i \in I, j \in J_i \\
		& x_{ij} - (1-F_i) \leq z_{ij} & \forall i \in I, j \in J_i \\
		& t_k - t_i \geq D_i + d_i^\omega G_i -\sum_{j \in J_i} D_i e_{ij} x_{ij} - \sum_{j \in J_i} d_i^\omega e_{ij} S_{ij} & \forall i \in I, (i,k) \in \mathcal{A}\\
		&\sum_{i \in I} \sum_{j \in J_i} b_{ij} x_{ij} \leq B &\\
		& \sum_{j \in J_i} x_{ij} \leq 1 & \forall i \in I\\
		& S_{ij} \leq G_i & \forall i \in I, j \in J_i \\
		& S_{ij} \leq x_{ij} & \forall i \in I, j \in J_i\\
		& S_{ij} \geq G_i + x_{ij} - 1 & \forall i \in I, j \in J_i \label{cons:s3}\\
		& z_{ij} = \hat{x}_{ij} & \forall i \in I, j \in J_i \label{cons:local1}\\
		& \tau_{i} = \hat{t}_{i} & \forall i \in I \label{cons:local2}\\
		& t_i \geq 0 & \forall i \in I \label{cons:tPos}\\
		& x_{ij} \in \{0,1\}. & \forall i \in I, j \in J_i\\
		& \tau_i \geq 0 & \forall i \in I \\
		& 0 \leq z_{ij} \leq 1 . & \forall i \in I, j \in J_i \label{cons:zRange}
		\end{align}
	\end{subequations}
	\begin{subequations}
		\label{prob:LagDual}
		\begin{align}
		(R^\omega) \quad \max_{\pi,\lambda} \min_{t,\tau,x,z} \quad & t_N + \sum_{i \in I} \pi_i(\tau_{i} - \hat{t}_i) + \sum_{i \in I} \sum_{j \in J_i} \lambda_{ij}(z_{ij} - \hat{x}_{ij}) & \\
		\text{s.t.} \quad & (\ref{cons:stN}) - (\ref{cons:s3})\text{ and } (\ref{cons:tPos}) - (\ref{cons:zRange}) \label{cons:X}
		\end{align}
	\end{subequations}
	The result from \cite{zou2016nested} could not be directly applied in our problem since our state variable contains both continuous variables \(t\) and binary variables \(x\). The Lagrangian cuts will only be a lower bound but the tightness could not be guaranteed. This means that if we carry out a cutting plane scheme using the Lagrangian cuts, solving the master problem (\ref{prob:master}) will only obtain a lower bound. \\
	\newline
	One important observation to help with this issue is that if we can determine the temporal relationship between every activity and the disruption, it is always optimal to start the activity at its earliest possible time. For a feasible solution \((\hat{t},\hat{x})\), we define the shifted solution \((\tilde{t},\tilde{x})\) of \((\hat{t},\hat{x})\) as:
	\begin{definition}
		Shifted solution of a feasible solution \((\hat{t},\hat{x})\):
		\begin{itemize}
			\item \(\tilde{x}_{ij} = \hat{x}_{ij}\)
			\item For all activities which do not have a predecessor,\\
				\(\tilde{t}_i = 
				\begin{cases}
					H^\omega & \text{if temporal relationship information specifies } t_i \geq H^\omega\\
					0 & \text{otherwise.}
				\end{cases}
				\)
			\item For all other activities, \\
				\(\tilde{t}_i = 
				\begin{cases}
				\max \{H^\omega, \max_{i' \in I, i' \in I_i} \{\tilde{t}_{i'} + D_{i'}(1 - \sum_{j \in J_{i'}} e_{i'j}\tilde{x}_{i'j})\} \} & \text{if the temporal relationship}\\
				& \text{information specifies } t_i \geq H^\omega\\
				\max_{i' \in I, i' \in I_i} \{\tilde{t}_{i'} + D_{i'}(1 - \sum_{j \in J_{i'}} e_{i'j}\tilde{x}_{i'j})\} & \text{otherwise.}
				\end{cases}
				\)
		\end{itemize}
	\end{definition}
	\noi We should notice that the temporal relationship should match the feasible solution. This is a natural result from solving problem~\ref{prob:subLocal} with temporal relationship information given. \\
	\newline
	It is obvious to see the following statement is true because delaying an activity as soon as possible when the crashing decision is decided and temporal relationship information is revealed will not bring any benefit.
	\begin{corollary}
		For an optimal solution \((t^*,x^*)\) with correct temporal relationship information given for all activities, its shifted solution is also optimal.
	\end{corollary}
	\noi Now we will prove that the Lagrangian cuts generated by \cite{zou2016nested} is tight and valid at any shifted solution if temporal relationship information is given for all activities. This statement is very important because it provides an accurate evaluation at the optimal solution, which will lead to an algorithm which is able to solve problem~\ref{prob:extensive} to the exactness.
	\begin{proposition} \label{prop:convex}
		Suppose the set defined by constraint (\ref{cons:X}) is denoted by \(X\). For any shifted solution \((\tilde{t},\tilde{x})\) with temporal relationship information given for all \( i \in I\), 
		\[conv(X) \cap \{(\tau,z) \mid \tau_i = \tilde{t}_i, z_{ij} = \tilde{x}_{ij}, \tau_i, z_{ij} \in \mathbb{R}, \forall i \in I, j \in J_i \}  = \]
		\[conv(X \cap \{(\tau,z) \mid \tau_i = \tilde{t}_i, z_{ij} = \tilde{x}_{ij}, \tau_i, z_{ij} \in \mathbb{R}, \forall i \in I, j \in J_i \} )\]
	\end{proposition}
	\begin{proof}
		Suppose \((t^0, \tau^0, x^0, z^0) \in  conv(X) \cap \{(\tau,z) \mid \tau_i = \tilde{t}_i, z_{ij} = \tilde{x}_{ij}, \tau_i, z_{ij} \in \mathbb{R}, \forall i \in I, j \in J_i \} \). Then there exist a set of points in \(X\) indexed by \(k \in \mathcal{K}\) where \((t^0, \tau^0, x^0, z^0) = \sum_{k \in \mathcal{K}}\rho_{k} (t^k, \tau^k, x^k, z^k) \) and \(\rho_{k} \geq, \sum_{k \in \mathcal{K}} \rho_{k} = 1\). Since \(\tilde{x}_{ij} \in \{0,1\}, z_{ij}^k \in [0,1], \forall k \in \mathcal{K}\) and we have \(\sum_{k \in \mathcal{K}} \rho_{k} z^k_{ij} = z^0_{ij} = \tilde{x}_{ij}\), every \(z^k_{ij}\) can only take the same value as \(\tilde{x}_{ij} , \forall k \in \mathcal{K}\), which is an end point of \([0,1]\).\\
		\newline
		\(z^k_{ij} = \tilde{x}_{ij}\) and temporal relationship information is given for all \(i \in I\), this means that the shifted solution \(\tilde{t}_i\) is the earliest possible starting time for activity \(i\), so \(\tau_i^k \geq \tilde{t}_i\). Similar to the previous statement about the relationship between \(z^k\) and \(\tilde{x}\), as \(\tilde{t}_i\) is taking an end point of the feasible set of \(\tau_i^k\) and \(\sum_{k \in \mathcal{K}} \rho_{k} \tau_i^k = \tilde{t}_i \), we have \(\tau_i^k = \tilde{t}_i,\ \forall k \in \mathcal{K}\). Combining the results above, 
		\[conv(X) \cap \{(\tau,z) \mid \tau_i = \tilde{t}_i, z_{ij} = \tilde{x}_{ij}, \tau_i, z_{ij} \in \mathbb{R}, \forall i \in I, j \in J_i \}  \subseteq \]
		\[conv(X \cap \{(\tau,z) \mid \tau_i = \tilde{t}_i, z_{ij} = \tilde{x}_{ij}, \tau_i, z_{ij} \in \mathbb{R}, \forall i \in I, j \in J_i \} )\]
		On the other hand, 
		\[conv(X) \cap \{(\tau,z) \mid \tau_i = \tilde{t}_i, z_{ij} = \tilde{x}_{ij}, \tau_i, z_{ij} \in \mathbb{R}, \forall i \in I, j \in J_i \}  \supseteq \]
		\[conv(X \cap \{(\tau,z) \mid \tau_i = \tilde{t}_i, z_{ij} = \tilde{x}_{ij}, \tau_i, z_{ij} \in \mathbb{R}, \forall i \in I, j \in J_i \} )\]
		is also true. Suppose a point \((t^0, \tau^0, x^0, z^0) \in conv(X \cap \{(\tau,z) \mid \tau_i = \tilde{t}_i, z_{ij} = \tilde{x}_{ij}, \tau_i, z_{ij} \in \mathbb{R}, \forall i \in I, j \in J_i \} )\), then we can find a set of points indexed by \(\mathcal{K}\) in \(X \cap \{(\tau,z) \mid \tau_i = \tilde{t}_i, z_{ij} = \tilde{x}_{ij}, \tau_i, z_{ij} \in \mathbb{R}, \forall i \in I, j \in J_i \} \) such that \((t^0, \tau^0, x^0, z^0) \) is the convex combination of them. For each \(k \in \mathcal{K}\), \(\tau_i^k = \tilde{t}_i\) and \(z_{ij}^k = \tilde{x}_{ij}\). Therefore, \(\tau^0 = \tilde{t}_i\) and \(z_{ij}^0 = \tilde{x}_{ij}\). Plus, each \((t^k, \tau^k, x^k, z^k) \in X \). Since \((t^0, \tau^0, x^0, z^0)\) is the convex combination of \((t^k, \tau^k, x^k, z^k)\), \((t^0, \tau^0, x^0, z^0) \in conv(X)\). Thus \((t^0, \tau^0, x^0, z^0) \in conv(X) \cap \{(\tau,z) \mid \tau_i = \tilde{t}_i, z_{ij} = \tilde{x}_{ij}, \tau_i, z_{ij} \in \mathbb{R}, \forall i \in I, j \in J_i \}  \).
	\end{proof}
	%the restricted version of the function \(f^\omega\) will be convex in \(t\), and we could find a tight and valid Lagrangian cut similar to those in \cite{zou2016nested}. This means, for example, if for a specific scenario \(\omega\), \(H^\omega = 5\), and for each activity \(i \in I\) we know whether its starting time \(t_i\) is less than, larger than or equal to \(H^\omega = 5\), then there is a finite set of linear constraints that could be generated to fully characterize \(f^\omega(x,t)\) in formulation~(\ref{prob:master}). This property is true mainly because once we know the temporal relationship between the starting time of each activity and the disruption time, it is always optimal to start the activity at its earliest possible time. We will prove this property as follows. 
	\noi With this property, the Lagrangian cuts derived in \cite{zou2016nested} are valid and tight at shifted solutions when temporal relationship information is given for all activities.
	\begin{theorem}
		For a specific scenario \(\omega\), let \(\theta^*\) be the objective value and \(\pi^*\) and \(\lambda^*\) be an optimal solution to the Lagrangian dual problem (\(R^\omega\)) in (\ref{prob:LagDual}), where the first stage solution is a shifted solution \((\tilde{t},\tilde{x})\) with temporal relationship information given for all \(i \in I\). The cut \(\theta^\omega \geq \sum_{i \in I} \pi_i^* (t_i - \tilde{t}_i) + \sum_{i \in I} \sum_{j \in J_i} \lambda_{ij}^* (x_{ij} - \tilde{x}_{ij}) + t_N^*\) is a valid and tight cut, where being valid and tight means:
		\begin{itemize}
			\item valid, \(f^\omega(x,t) \geq \sum_{i \in I} \pi_i^* (t_i - \tilde{t}_i) + \sum_{i \in I} \sum_{j \in J_i} \lambda_{ij}^* (x_{ij} - \tilde{x}_{ij}) + t_N^*,\ \forall t_i \geq 0, x_{ij} \in \{0,1\}\);
			\item tight, \(f^\omega(\tilde{x},\tilde{t}) = \theta^*\).
		\end{itemize} 
	\end{theorem}
	\begin{proof}
		The Lagrangian cut is a lower bound at any solution because of it is a relaxation of problem~(\ref{prob:subLocal}), which means it is always valid. See Proposition 6.1 in \cite{nemhauser1988integer} for the detailed proof.\\
		\newline
		From Proposition~\ref{prop:convex} and Corollary 6.5 from \cite{nemhauser1988integer}, we know the objective value of the Lagrangian dual problem equals to the objective value of the original integer programming problem.
	\end{proof}
	\noi The temporal relationship could be determined by a branch-and-bound scheme. Although theoretically for each scenario there are exponential number of possible temporal relationships because each activity can either be before the disruption time or after that, there are only a small number of activities, of which the optimal starting time of the deterministic problem (\ref{prob:static}) is close to the disruption time, that really matter. If an activity should be starting much earlier than the disruption time in the deterministic problem, it usually will significantly increase the length of the project span by delaying the start of such activity. If an activity cannot start until a time much later than the disruption time, it is usually not possible to make it start before the disruption time without committing crashing resources heavily on its precedent activities, which may likely lead to an inferior solution.\\
	\newline
	We now present a branch-and-cut algorithm that generates both Lagrangian cuts and shifted cuts:
	\begin{algorithm}[H]
		\caption{Branch-and-cut procedure for the project crashing problem with one stochastic disruption}
		\label{alg:Cut}
		\begin{algorithmic}[1]
			\State Initialize with an empty node list. Create a node \(0\) and append it to the node list. Let \(LB = -\infty\) and \(UB = +\infty\) denote the lower bound and upper bound of the entire branching tree.
			\State At node \(0\), initialize with a master problem \((M)\) and a set of sub problems \((S^\omega)\). Let \(k = 0\) be the index of nodes. Let \(c_m\) denote the number of iterations that the lower bound \(LB\) has not been improved.
			\While{\(LB < UB\) and \(c_m \leq m\)}
			\For{\(\omega \in \Omega\)}
				\State aaa
			\EndFor{\textbf{end for}}
			
			\EndWhile{\textbf{end while}}
			
			\State Plug \(\hat{s}^{p,k}, \hat{s}^{q,k}\) in the sub problem \(SDI\) and solve the sub problem. Record solution \(\lambda^{p,k}, \lambda^{q,k}\) and optimal objective value \(z_{feas}^k\);
			\While{\(z_{feas}^k > \epsilon\)} 
			\State Generate the feasibility cut as \(z_{feas}^k -\lambda^{p,k}(s^p - \hat{s}^{p,k}) - \lambda^{q,k} (s^q - \hat{s}^{q,k}) \leq 0\);
			\State Append the feasibility cut to the master problem and obtain the updated master problem as \((M^{k+1})\);
			\State \(k := k+1\);
			\State Solve the master problem \((M^k+1)\); \\
			\State Record solution \(\hat{s}^{p,k}, \hat{s}^{q,k}\) and optimal objective value \(LB = v^*(M^k)\); 
			\State Plug \(\hat{s}^{p,k}, \hat{s}^{q,k}\) in the sub problem \(SDI\) and solve the sub problem; \\
			\State Record solution \(\lambda^{p,k}, \lambda^{q,k}\) and optimal objective value \(z_{feas}^k\);
			\EndWhile{\textbf{end while}}
			\State Output \(LB\) as the lower bound of the robust ACOPF problem and \(\hat{s}^{p,k}, \hat{s}^{q,k}\) as the \(\epsilon\)-feasible solution.
		\end{algorithmic}
	\end{algorithm}
	\noi We prove that when this algorithm terminates, the solution obtained will be optimal because of the tightness property described before in this section.
	\begin{theorem}
		Algorithm~\ref{alg:Cut} will terminate with an optimal solution to model~\eqref{prob:extensive}.
	\end{theorem}
	\begin{proof}
		aaa
	\end{proof}
\section{Computational Results} \label{sec:results}
	\subsection{Computational Performance}
		With the implementation of Algorithm~\ref{alg:Cut}, we aim to test the computational performance of the algorithm with a group of test data, where the following performance measures will be specifically checked:
		\begin{itemize}
			\item How many nodes does the algorithm explore?
			\item How many cuts does the algorithm generate before reaching optimality?
			\item How fast do the upper bound and the lower bound converge?
		\end{itemize}
		\textcolor{blue}{Include the test results here.}
	\subsection{Benchmark Test against Deterministic Project Crashing Problem}
		We will also compare the result of our model and a deterministic project crashing problem with a known disruption time and a known disruption magnitude as the benchmark, in order to justify the value of our model. We suppose the random variable for the timing of the stochastic disruption is denoted by \(H\) and the random variable for the duration of each activity \(i \in I\) after the disruption is denoted by \(\tilde{D}_i\). We also suppose that the mean values of those random variables are known as \(\E[H]\) and \(\E[\tilde{D}_i],\ i \in I\). A deterministic, moderate-sized mixed integer program can be established as follows: 
		\begin{subequations}
			\label{prob:dextensive}
			\begin{align}
			\min \quad & t_N &\\
			\text{s.t.} \quad & t_N \geq t_i & \forall i \in I \label{cons:dtN}\\
			& \E[H] - F_i M \leq t_i & \forall i \in I \label{cons:dFt}\\
			& \E[H] + G_i M \geq t_i & \forall i \in I \label{cons:dGt}\\
			& F_i + G_i = 1 & \forall i \in I \label{cons:dFG}\\
			& t_k - t_i \geq D_i F_i + \E[\tilde{D}_i] G_i - \sum_{j \in J_i} D_i e_{ij} x^F_{ij} - \sum_{j \in J_i} \E[\tilde{D}_i] e_{ij} x^G_{ij} & \forall i \in I, (i,k) \in \mathcal{A} \label{cons:dtt}\\
			& x_{ij} \geq x_{ij}^F & \forall i \in I,\ j \in J_i \label{cons:dxF1}\\
			& F_i \geq x_{ij}^F & \forall i \in I,\ j \in J_i \label{cons:dxF2}\\
			& x_{ij} + F_i - 1 \leq x_{ij}^F & \forall i \in I, \ j \in J_i \label{cons:dxF3}\\
			& x_{ij} \geq x_{ij}^G & \forall i \in I,\ j \in J_i \label{cons:dxG1}\\
			& G_i \geq x_{ij}^G & \forall i \in I,\ j \in J_i \label{cons:dxG2}\\
			& x_{ij} + G_i - 1 \leq x_{ij}^G & \forall i \in I, \ j \in J_i \label{cons:dxG3}\\
			& \sum_{i \in I} \sum_{j \in J_i} b_{ij} x_{ij} \leq B & \label{cons:dBudget}\\
			& \sum_{j \in J_i} x_{ij}\leq 1 & \forall i \in I \label{cons:dOneCrash}\\ 
			& t_i \geq 0 & \forall i \in I\\
			& F_i, G_i \in \{0,1\} & \forall i \in I \\
			& x_{ij}, x_{ij}^F, x_{ij}^G \in \{0,1\} & \forall i \in I.
			\end{align}
		\end{subequations}
		The objective of this model is to minimize the total project span, assuming that the disruption happens at the expected time \(\E[H]\) and the disruption magnitude will take its mean value. Constraints~\eqref{cons:dtN} - \eqref{cons:dFG} replicate constraints~\eqref{cons:tN} - \eqref{cons:FG}. Constraint~\eqref{cons:dtt} determines whether the duration of each activity \(i \in I\) takes its nominal value \(D_i\) or its after-disruption value \(\E[\tilde{D}_i]\). Constraints~\eqref{cons:dxF1} - \eqref{cons:dxG3} linearize bilinear terms \(x_{ij} F_i\) and \(x_{ij}G_i\). The budget constraint~\eqref{cons:dBudget} and the constraint of only one crashing per activity~\eqref{cons:dOneCrash} are inherited as well. \\
		\newline 
		Apparently this is a simplified version of our disruption model because both the timing and the magnitude of the disruption is assumed deterministic in this case. If the gap between the objective value of model~\eqref{prob:dextensive} and that of model~\eqref{prob:extensive} is large, we can conclude that our model adds a significant value by including the stochastic disruption. We can also fix only the timing or the magnitude of the disruption and run similar tests. This way we will be able to tell which factor, timing or magnitude, is more important to be modeled as a random variable.\\
		\newline
		\textcolor{blue}{Include the test results here.}

\section{Conclusions} \label{sec:conclusions}
	\begin{itemize}
		\item Contributions
		\item Further directions
	\end{itemize}

\bibliographystyle{plain}
\bibliography{PERT_Bib}

\end{document}